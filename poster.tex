\documentclass[final, xcolor=cmyk]{beamer}
\mode<presentation>
\usepackage{lipsum} %just for random txt 
\usepackage{graphicx}


%\usepackage[UTF8]{ctex} %if you want to input Chinese/Japanese, un-comment this line


% STEP 0: Choose your program
% Applied Psychology:               AppliedPsychology
% CST:                              ComputerScienceTechnology
% DS:                               DataScience
% Financial Mathematics:            FinancialMathematics
% Food Science:                     FoodScience 
% Environmental Science:            EnvironmentalScience
% Statistics:                       Statistics
\setbeamercolor{frame title}{fg=black,bg=Statistics}

% STEP 1: Change the next line according to your language
\usepackage[english]{babel}

% STEP 2: Make sure this character encoding matches the one you save this file as
% (this template is utf8 by default but your editor may change it, causing problems)
\usepackage[utf8x]{inputenc}

% You probably don't need to touch the following four lines
\usepackage[T1]{fontenc}
\usepackage{lmodern}
\usepackage{amsmath,amsthm, amssymb, latexsym}
\usepackage{calc}
\usepackage{exscale} % required to scale math fonts properly
\usepackage{eulervm} % Euler VM symbols
%\usepackage[cmbrightmath,scaleupmath]{tpslifonts}
%\usepackage{cmbright} % CM Bright math
\usepackage{ragged2e} 
\usepackage{svg}
\graphicspath{{img/}}
%\usepackage[orientation=portrait,size=a0,scale=1.3]{beamerposter}
\usepackage[orientation=portrait,size=custom,width=90,height=140,scale=0.95]{beamerposter}

\usepackage{tikz}
\newcommand*\circled[1]{\tikz[baseline=(char.base)]{
        \node[shape=circle,draw,inner sep=2pt] (char) {#1};}}

% STEP 3:
% Change colours by setting \usetheme[<id>, twocolumn]{UIC}.
\usetheme[color, twocolumn]{UIC}

%\setbeamertemplate{background}[grid]{}

% STEP 4: Set up the title and author info
\titlestart{
    \begin{center}
    How to Make a Academic Poster for United International College DST
    \end{center}
} % first line of title

\author[*]{
    \begin{center}
        Dr.John, Smith, \ Prof.Trump.Johnson
    \end{center}
}% author & supervisor
\titlesize{\Huge} % Use this to change title size if necessary

\institute[*]{
    \centering
    Statistics Program, Division of Science and Technology, BNU-HKBU United International College \\
    Corresponding student author. Tel: +86-131-12345678, E-mail:\href{mailto:exmaple@example.com}{JohnS77@mail.uic.edu.hk} \\
    Corresponding author. Tel: +85-136-87654321, E-mail: \href{mailto:exmaple@example.com}{MAGA@uic.edu.hk}

}


% Stuff such as logos of contributing institutes can be put in the lower left corner using this
%\leftcorner{}

%\newcommand{\figfont}{\normalsize} % set fotsize for figures 

\begin{document}
\begin{poster}
%%%%%%%%%%%%%%%%%%%%%%%%%%%%%%%%%%%%%%%%%%%%%%%%%%%%%%%%%%%%%%%%%%%%%%%%%%%%%%
% First column %%%%%%%%%%%%%%%%%%%%%%%%%%%%%%%%%%%%%%%%%%%%%%%%%%%%%%%%%%%%%%%
\newcolumn

\section{Motivation}
\justifying
\begin{itemize} 
    \justifying
    \item The AR does not provide us a good template 
    \item Everyone who study in DST need to know \LaTeX \cite{mittelbach2004latex}
\end{itemize}
% change line spacing

\lipsum[1-2]

\vspace*{5ex}
\section{How to Use this Template} \justifying
The First thing we need to know what is \LaTeX 
\vspace{3ex}
\begin{columns}
    \figfont
    \newcommand{\figwidth}{\columnwidth}
    \begin{column}{\columnwidth}
        \includesvg[clean, width=0.7\columnwidth]{img/IP_band_edge2}
        \centering{
        \caption{Band diagrams of InAs and \\ GaAs at equilibrium}
        }
    \end{column}
\end{columns}
\lipsum[2]

\vspace{5ex}

\section{Simulation} \justifying
%Capacitance-voltage characteristics were calculated from potential distributions obtained by solving Poisson equation with modified Thomas-Fermi approximation (MTFA).   

Poisson equation with modified Thomas-Fermi approximation (MTFA) was used to calculate CV characteristics. \cite{batts2007beamer}
\vspace*{-1ex}

\subsection{Poisson equation}
\vspace*{-1ex}
    $$
        \frac{d^2\varphi}{dz^2} =
         -\frac{q}{\varepsilon\varepsilon_0}\left[N_D^+ - N_A^- - n(z) + \
          p(z)\right] 
    $$
    
    with electron concentration:
    $$
        n(z) = \int_{0}^{\infty}\rho_c(z,E) f_{FD}(E) f_{MTFA}(z,E)dE
    $$ 
    
    and  DOS for non-parabolic conduction band:  
    $$
        \rho \left(z,E\right) = \frac{1}{2\pi^2} \left(\frac{2m_{e}}{\hbar^2}\right)^{3/2} \!\!\! \sqrt{E} \cdot \sqrt{1+\alpha E} \cdot \left(1+ 2\alpha E \right)
    $$  
    
%    $$
%   \text{where}\ \alpha = \frac{1}{E_g}\left(1 - \frac{m_{\Gamma}}{m_0}\right)^{\!2}\ \ \text{--- nonparabolicity coefficient}
%    $$

\vspace{-1ex}       
\subsection{Modified Thomas-Fermi approximation}
    MTFA used to take into account boundary condition for wave function during accumulation.  
    $$
    f_{MTFA}(z, E)  = 1 - sinc\left( \frac{2z}{L} \left(\frac{E}{k_BT}\right)^{1/2} \left(1+\alpha E\right)^{1/2}\right)
    $$
    
%%%%%%%%%%%%%%%%%%%%%%%%%%%%%%%%%%%%%%%%%%%%%%%%%%%%%%%%%%%%%%%%%%%%%%%%%%%%%%
% Second column %%%%%%%%%%%%%%%%%%%%%%%%%%%%%%%%%%%%%%%%%%%%%%%%%%%%%%%%%%%%%%
\newcolumn
\section{Results}

\lipsum[1-3]

\begin{columns}
    \figfont
    \newcommand{\figwidth}{\columnwidth}
    \begin{column}{\columnwidth}
        \includesvg[clean, width=0.7\columnwidth]{img/IP_band_edge1}
        \centering{
        \caption{Test Figure Two}
        }
    \end{column}
\end{columns}

\lipsum[4-10]

\section{Reference}
\footnotesize
\bibliographystyle{IEEEtran}
\bibliography{reference}


\end{poster}
\end{document}